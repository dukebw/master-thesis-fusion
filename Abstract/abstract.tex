% ************************** Thesis Abstract *****************************
% Use `abstract' as an option in the document class to print only the titlepage and the abstract.
In this thesis by articles we make contributions related to attention and
fusion at the intersection of the deep learning and computer vision fields.

In our first article, we investigated the design of neural network operators
that fuse features extracted from different modalities, such as audio and
video, and use the fused representation to make predictions on a single task,
such as visual question answering (VQA) or activity recognition.
We propose a generalized class of multimodal fusion operators for the task of
VQA.
We identify generalizations of existing multimodal fusion operators based on
the Hadamard product, and show that specific non-trivial instantiations of this
generalized fusion operator exhibit superior performance in terms of open-ended
accuracy on the VQA task.
In particular, we introduce Nonlinearity Ensembling, Feature Gating, and
post-fusion neural network layers as fusion operator components, culminating in
an absolute percentage point improvement of~$1.1\%$ on the VQA~2.0 test-dev set
over baseline fusion operators, which use the same features as input.
We use our findings as evidence that our generalized class of fusion operators
could lead to the discovery of even superior task-specific operators when used
as a search space in an architecture search over fusion operators.

In our second article, we introduced Transformers, which are an attention-based
architecture, to the video object segmentation (VOS) task.
VOS requires tracking an object through a video
with pixel-level accuracy, and has previously been solved either through
combinations of online fine-tuning or recurrent feature propagation, which have
inherent disadvantages.
Online fine-tuning methods have slow runtimes, while recurrent methods lack
scalability and have the drawback of compounding error, due to their sequential
nature.
We propose a scalable, end-to-end method for VOS called
``Sparse Spatiotemporal Transformers'' (SST) to address said runtime,
scalability and temporal dependency issues.
SST extracts per-pixel representations for each object by simultaneously
attending to high semantic information feature vectors at each pixel in a
video.
SST performs inference in a single feedforward pass of a Transformer
architecture designed with sparse attention components.
We contribute the first method for using attention-based networks for VOS, and
demonstrate the scalability advantage of attention-based over recurrent
networks in the spatiotemporal domain.
We show that our attention-based feedforward method achieves competitive
results on YouTube-VOS 2019 at a fraction of the runtime of state of the art
approaches.
% TODO(brendan): runtime + scalability numbers from on-device profiling.
